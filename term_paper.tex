\documentclass[12pt, a4paper, oneside]{article}
\usepackage[utf8]{inputenc}
\usepackage[english, russian]{babel}
\usepackage[T1, T2A]{fontenc}
\usepackage{titlesec}
\usepackage{hyperref}
\usepackage{tikz}
\titleformat{\section}[block]{\color{black}\Large\bfseries\filcenter}{}{1cm}{}
\titleformat{\subsection}[hang]{\bfseries\filcenter}{}{1cm}{}
\setcounter{secnumdepth}{0}
\hypersetup{
    colorlinks=true,
    linkcolor=black,
    urlcolor=blue,
}
\newcommand{\ExternalLink}{
    \tikz[x=1.2ex, y=1.2ex, baseline=-0.05ex]{
        \begin{scope}[x=1ex, y=1ex]
            \clip (-0.1,-0.1) 
                --++ (-0, 1.2) 
                --++ (0.6, 0) 
                --++ (0, -0.6) 
                --++ (0.6, 0) 
                --++ (0, -1);
            \path[draw, 
                line width = 0.5, 
                rounded corners=0.5] 
                (0,0) rectangle (1,1);
        \end{scope}
        \path[draw, line width = 0.5] (0.5, 0.5) 
            -- (1, 1);
        \path[draw, line width = 0.5] (0.6, 1) 
            -- (1, 1) -- (1, 0.6);
        }
    }

\title{Курсовая работа}
\author{Абдуллаева Евгения Гасановна}
\date{Май 2020}

\begin{document}

    \begin{titlepage}
        \begin{center}
            Филиал Московского Государственного Университета\\
            имени М.В. Ломоносова в городе Ташкенте\\
            \vspace{0.5cm}
            Факультет прикладной математики и информатики\\
            \vspace{0.5cm}
            Кафедра прикладной математики и информатики\\
            \vfill
            Абдуллаева Евгения Гасановна\\
            \vfill
            \textbf{\MakeUppercase{Курсовая работа}\\
            на тему: «Тема курсовой работы»\\
            \vspace{0.5cm}
            по направлению 01.03.02 «Прикладная математика и информатика»}
        \end{center}
        \vfill
        \begin{flushleft}
            Научный руководитель\\
            к.ф.-м.н., в.н.с. Алисейчик Павел Александрович
        \end{flushleft}
        \vspace{0.5cm}
        \begin{flushright}
            «$\rule{1cm}{0.15mm}$» $\rule{3cm}{0.15mm}$ 2020 г.
        \end{flushright}
        \vfill
        \begin{center}
            Ташкент 2020 г.
        \end{center}
    \end{titlepage}
    
    \section*{Тема курсовой работы}
        \paragraph{}
        Аннотация на русском
        \paragraph{}
        Abstract in English (не нужно)
    \setcounter{page}{2}
    \newpage
    
    \tableofcontents
    \newpage
    
    \section*{Введение}
    \addcontentsline{toc}{section}{Введение}
        \paragraph{} % Постановка проблемы, цель работы
        За годы сопровождения курса «Практикум на ЭВМ» системой «Форум МГУ» была накоплена база сообщений, включающая вопросы студентов и ответы преподавателей по различным этапам и темам данного курса, являющаяся ценным образовательным ресурсом. В связи с созданием системы «МГУ Контест» возникла необходимость переноса базы сообщений в новую систему. Сохранение накопленной базы сообщений ведет как к экономии времени преподавателей, освобождая их от необходимости многократно отвечать на повторяющиеся вопросы, так и студентов, которым не придется ожидать ответа на заданный вопрос, а следовательно способствует развитию обучающей системы.
        \paragraph{} % Актуальность
        В настоящий момент качество образования является одним из ключевых факторов определяющих развитие общества, именно благодаря образованию человек приобретает способность свободно и независимо мыслить, имплементировать собственные идеи и эффективно использовать интеллектуальные ресурсы для решения множества практических задач. Преподавание в наши дни представляет собой нить, по которой ученик продвигается посредством исследований и открытий. Следовательно, актуальной задачей является повышение эффективности процесса обучения, а в частности помощь преподавателям в предоставлении студентам обучающего материала.
    \newpage
    
    \section{Исследование структуры баз данных систем}
        Текст про структуру БД систем «Форум МГУ» и «МГУ Контест»
    \newpage
    
    \section{Методы и организация работы}
    Текст про Python/Django, MySQL
    \newpage
    
    \section{Подготовка к переносу сообщений}
        \subsection{Выделение досок необходимых для переноса}
            \paragraph{}
            Первая задача, которую необходимо выполнить для подготовки к переносу сообщений – определить, какие сообщения следует переносить, так как обязательно соответствие между темами, обсуждаемыми в сообщениях, и содержанием курсов системы «МГУ Контест».
            \paragraph{}
            База данных системы «Форум МГУ» содержит следующие доски:
            \begin{enumerate}
                \item Практикум на ЭВМ: общие положения
                \item Практикум по программированию, 1 курс, 1 семестр
                \item Практикум по программированию, 1 курс, 2 семестр
                \item Практикум по программированию, 2 курс, 1 семестр
                \item Практикум по программированию, 2 курс, 2 семестр
                \item Практикум по программированию, 3 курс, 1 семестр
                \item Практикум по программированию, 3 курс, 2 семестр
                \item Практикум по программированию, 4 курс, 1 семестр
                \item 4 курс, 2 семестр
                \item Магистратура мех-мат ф-та, практикум
                \item Прочее
                \item Ассемблер
                \item Семинар «Программирование интеллектуальных систем»
                \item Системы программирования
                \item Спортивное программирование
                \item Курсовые и дипломные работы
                \item Служебный раздел
            \end{enumerate}
            \paragraph{}
            База данных системы «МГУ Контест» содержит курсы, соответствующие этапам «Практикума на ЭВМ» начиная от первого семестра первого курса до первого семестра четвертого курса, перечисленные ниже:
            \begin{enumerate}
                \item Простые алгоритмы
                \item Дискретная математика
                \item Операционные системы
                \item Основы ООП
                \item Дискретная оптимизация
                \item Численные методы, часть 1
                \item Численные методы, часть 2
            \end{enumerate}
            \paragraph{}
            Перенос осуществляется лишь для тех сообщений из базы данных системы «Форум МГУ», которые принадлежат доскам, соответствующим этапам курса «Практикум на ЭВМ» или содержат общие положения проведения «Практикума на ЭВМ».
            \paragraph{}
            Средства управления администратора, предоставляемые Django, не позволяют просматривать текст нескольких сообщений, сгруппированных по какому-либо признаку, поэтому для просмотра сообщений из базы данных системы «Форум МГУ» с указанием их принадлежности к определенной доске используется следующее решение.
            \paragraph{}
            Реализована функция \texttt{get\_messages\_view}, которая позволяет получить представление списка сообщений, разделенных по доскам и темам в виде текста с HTML-форматированием. Данная функция принимает в качестве аргументов список досок \texttt{all\_boards}, список тем \texttt{all\_topics} и список сообщений \texttt{all\_messages}, которые необходимо отобразить. В теле функции объявляется переменная \texttt{html} типа \texttt{str}, изначально пустая. Для каждой доски из списка досок \texttt{all\_boards} к переменной \texttt{html} прибавляется идентификатор и имя доски, заключенные в HTML-теги заголовка первого уровня; переменной \texttt{topics} присваивается список тем из \texttt{all\_topics}, принадлежащих рассматриваемой в текущий момент доске. Для каждой темы из \texttt{topics} к переменной \texttt{html} прибавляется идентификатор и название темы (определяется как заголовок первого сообщения в теме), заключенные в HTML-теги заголовка второго уровня; переменной \texttt{messages} присваивается список сообщений из \texttt{all\_messages}, относящихся к рассматриваемой в текущий момент теме; сообщения в списке \texttt{messages} сортируются по возрастанию даты их написания, определяемой полем \texttt{postertime} модели сообщения. Для каждого сообщения из \texttt{messages} к переменной \texttt{html} прибавляется идентификатор, заголовок, имя автора сообщения, имя изменявшего текст сообщения пользователя, дата написания и непосредственно текст сообщения. Функция возвращает переменную \texttt{html}.
            \paragraph{}
            Описанная функция вызывается как возвращаемое значение function-based view (представлении в функциональном виде) \texttt{all\_messages}. В качестве аргументов в функцию передаются списки всех существующих в базе досок, всех тем и всех сообщений. Таким образом данное view предоставляет возможность просмотра сообщений, сгруппированных по принадлежности к доскам и темам в браузере по URL-адресу \texttt{<host>:<port>/all\_messages}.
            \paragraph{}
            Выделение необходимых для переноса досок осуществляется путем изучения содержания сообщений, принадлежащих этим доскам.
    \newpage
    
    \section{Перенос сообщений}
    \newpage
    
    \section{Заключение}
    Выводы
    \newpage
    
    \begin{thebibliography}{3}
    \addcontentsline{toc}{section}{Список литературы}
        \bibitem{}
            Документация Python\\
            \url{https://docs.python.org/3}
        \bibitem{}
            Документация Django\\
            \url{https://docs.djangoproject.com/en/3.0}
        \bibitem{}
            Документация MySQL\\
            \url{https://dev.mysql.com/doc/}
    \end{thebibliography}
    \newpage
    
    \section*{Приложения}
    \addcontentsline{toc}{section}{Приложения}
        \begin{itemize}
            \item[\ExternalLink] Инструменты для работы с базой данных системы «Форум МГУ»\\
            \url{https://github.com/eugeuie/forum}
            \item[\ExternalLink] Репозиторий проекта «МГУ Контест»\\
            \url{https://github.com/ruslanbektashev/contest}
        \end{itemize}

\end{document}
